% --- アブストラクト ---
\usepackage[addtotoc]{abstract}
% --- 文章の装飾・強調 ---
\usepackage[normalem]{ulem} % 取り消し線、下線などの装飾
\usepackage{appendix} % 付録の設定
\usepackage{pifont} % チェックマークなどの記号

% --- 数式関連 ---
\usepackage{amsmath, amssymb} % 数式の拡張
\usepackage{amsthm} % 定理環境の拡張
\usepackage{bm} % 太字ベクトル
\usepackage{cases} % 数式の分岐

% --- マクロ・レイアウト ---
\usepackage{ascmac} % 枠囲み、吹き出しなどのデザイン
\usepackage{fancyhdr} % ヘッダ・フッタのカスタマイズ
\usepackage{multicol} % 複数列のレイアウト
\usepackage[section]{placeins} % セクションごとにFloatBarrierを適用
\usepackage{lipsum} %twocolumn
\usepackage{lineno} % 行番号

% --- 色の設定 ---
\usepackage[dvipsnames, svgnames, table]{xcolor}

\usepackage{hyperref} % ハイパーリンク(必要なら解除)

% --- 画像・図表 ---
\usepackage{graphicx} % 画像の挿入
\usepackage{subcaption} % 図のサブキャプション
\captionsetup{compatibility=false} % キャプションの互換性設定
\usepackage{here} %Hオプション
% --- キャプション設定(回避策) ---
\makeatletter
\let\MYcaption\@makecaption
\makeatother
\makeatletter
\let\@makecaption\MYcaption
\makeatother

% --- コードの挿入 ---
\usepackage{listings}
\lstdefinestyle{tmy}{
    backgroundcolor=\color{bgcolor},
    frame=lines,
    framesep=2mm,
    basicstyle=\ttfamily\small,
    numbers=left,
    numbersep=6pt,
    breaklines=true,
    breakatwhitespace=true,
    captionpos=b,
    keywordstyle=\color{Blue}\bfseries,
    commentstyle=\color{Gray},
    stringstyle=\color{Maroon}
}
\lstset{style=tmy}

\usepackage[newfloat]{minted}
\captionsetup[listing]{position=bottom}
% 背景色を指定
\definecolor{bgcolor}{rgb}{0.95,0.95,0.92}

% mintedパッケージの設定
\setminted{
    bgcolor=bgcolor,        % 背景色を適用
    frame=lines,            % 枠線を表示
    framesep=2mm,           % コードと枠線の間隔
    fontsize=\small,        % フォントサイズ
    linenos=true,           % 行番号を表示
    breaklines=true,        % 行を折り返す
    breakanywhere=true,     % どこでも改行可能に
    breakautoindent=true    % 自動インデント
}

% --- アニメーション ---
\usepackage{animate} % アニメーションの挿入